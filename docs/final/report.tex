\documentclass[conference]{IEEEtran}
\IEEEoverridecommandlockouts

\usepackage{cite}
\usepackage[portuges,brazil,english]{babel}
\usepackage{amsmath,amssymb,amsfonts}
\usepackage{algorithmic}
\usepackage{graphicx}
\usepackage{textcomp}
\usepackage[latin1,utf8]{inputenc}
\usepackage{listings}
\usepackage{color}

% Commands to help in references
\newcommand{\rchap}[1]{Chapter~\ref{chap:#1}}
\newcommand{\rsec}[1]{Section~\ref{sec:#1}}
\newcommand{\rsecs}[2]{Sections~\ref{sec:#1} --~\ref{sec:#2}}
\newcommand{\rtab}[1]{Tabela~\ref{tab:#1}}
\newcommand{\rfig}[1]{Figura~\ref{fig:#1}}
\newcommand{\rfigs}[2]{Figures~\ref{fig:#1} --~\ref{fig:#2}}
\newcommand{\rfign}[3]{Figures~\ref{fig:#1}, \ref{fig:#2} \& \ref{fig:#3}}
\newcommand{\rlst}[1]{Listing~\ref{lst:#1}}
\newcommand{\rlsts}[2]{Listing~\ref{lst:#1}~--~\ref{lst:#2}}
\newcommand{\rlstn}[3]{Listings~\ref{lst:#1}{#2}~--~\ref{lst:#1}{#3}}
\newcommand{\req}[1]{Equation~\ref{eq:#1}}
\newcommand{\reqs}[2]{Equations~\ref{eq:#1} --~\ref{eq:#2}}
\newcommand{\ttt}[1]{{\texttt{#1}}}
\newcommand{\tbt}[1]{{\textbf{#1}}}
\newcommand{\tit}[1]{{\textit{#1}}}
\newcommand{\ts}{\textsuperscript}

\def\BibTeX{{\rm B\kern-.05em{\sc i\kern-.025em b}\kern-.08em
    T\kern-.1667em\lower.7ex\hbox{E}\kern-.125emX}}

\definecolor{codegreen}{rgb}{0,0.6,0}
\definecolor{codegray}{rgb}{0.5,0.5,0.5}
\definecolor{codepurple}{rgb}{0.58,0,0.82}
\definecolor{backcolour}{rgb}{0.95,0.95,0.92}
\definecolor{darkblue}{rgb}{0.0,0.0,0.6}

\lstdefinestyle{mystyle}{
  commentstyle=\foonotesize\color{codegreen},
  backgroundcolor=\color{backcolour},
  stringstyle=\color{codepurple},
  basicstyle=\footnotesize,
  breakatwhitespace=false,
  breaklines=true,
  captionpos=b,
  keepspaces=true,
  showspaces=false,
  showstringspaces=false,
  showtabs=false,
  tabsize=2
}

\lstset{
  classoffset=0,
  keywordstyle=\color{back},
  classoffset=1,
  morekeywords={use,hrw,module,increment,HARP,Catapult,synthesize},
  keywordstyle=\color{purple},
  classoffset=2,
  morekeywords={pragma,omp,parallel,target,map,data,device,for,},
  keywordstyle=\color{darkblue},
  frame=single,
  style=mystyle
}

\begin{document}

\title{Desmistificando o Cartola FC}

\author
{
  \IEEEauthorblockN{Ciro Ceissler}
  \IEEEauthorblockA{RA 108786\\ ciro.ceissler@gmail.com}
  \and
  \IEEEauthorblockN{Lucas de Souza e Silva}
  \IEEEauthorblockA{RA 140765\\ lucasonline1@gmail.com}
  \and
  \IEEEauthorblockN{Matheus Laborão Netto}
  \IEEEauthorblockA{RA 137019\\ mln.laborao@gmail.com}
  \and
  \IEEEauthorblockN{Ramon Nepomuceno}
  \IEEEauthorblockA{RA 192771\\ ramonn76@gmail.com}
}

\maketitle

\begin{abstract}

TODO:

\end{abstract}

\section{Introdução}

O futebol é um esporte com diversos  fãs ao redor do mundo e com uma
imprevisibilidade muito grande, supreendendo os torcedores. Um exemplo
recente foi  a Copa  do Mundo de  2018 no qual  o campeão  do torneio
anterior não conseguiu passar nem da primeira fase, perdendo 2x0 para
Coreia do Sul apenas 57\ts{a}  colocada no ranking FIFA \cite{fifa}, e
a presença  Croácia na  final. Ainda nesta  Copa do  Mundo; diversos
bancos, incluindo  o \tit{Goldman Sachs}, utilizaram  este evento para
demonstrar a capacidade de  prever eventos complexos \cite{news}, eles
chegaram a  rodar milhões de  variações do torneio para  calcular a
probabilidade de cada time avançar na competição e mesmo assim não
obtiveram um resultado satisfatório.

Os fãs  de futebol  também participam  da "brincadeira"  através do
Cartola FC, um  "fantasy game" sobre este esporte. O  Cartola FC é um
jogo online fictício no qual você pode montar seu time com jogadores
reais da Série A do Campeonato  Brasileiro. No jogo é preciso montar
seu time, escolhendo  técnico, jogadores e o esquema  tático. Com as
moedas  do  jogo, inicialmente  C\$  100.00  cartoletas, é  possível
escalar  o seu  time, comprando  e vendendo  jogadores a  cada rodada,
importante  adicionar que  a  (des)valorização  do jogador  acontece
após cada  rodada e leva  em consideração a pontuação  do atleta,
além da  média dos  outros jogadores. A  escalação deve  ser feita
antes do  primeiro jogo da rodada  e a cada uma,  os jogadores recebem
pontuações  baseadas  em suas  ações  durante  as partidas,  e.g.,
gols feitos,  chutes defendidos,  bolas roubadas, faltas  cometidas ou
sofridas, cartões  recebidos, entre outras. As  ações dos jogadores
em campo são  chamadas de Scouts e são elas  que geram a pontuação
do time. Por fim, o Capitão tem sua pontuação duplicada e não pode
ser o técnico.

No  Cartola  FC   de  2017,  o  jogador   vencedor  totalizou  2460.19
pontos  ao final  de  todas  as rodadas  (média  de  67.4 pontos  por
rodada),  competindo   contra  4  milhões  de   times  escalados.  Em
\cite{viscondiaplicacao}, um sistema  de previsão aplicando técnicas
de  \tit{machine  learning}  foi  proposto  para  conseguir  a  melhor
escalação do time a em cada rodada, entretanto os resultados obtidos
não permitiria vencer  o torneio. Além disso, este  trabalho fez uma
análise dos  dados, explanando algumas observações  como: o esquema
tático tem uma leve variação, sendo o 4-4-2 a melhor formação.

O projeto propõe um modelo  preditivo para maximizar a pontuação do
Cartola FC, informando  a cada rodada do jogo  quais parâmetros devem
ser utilizados, ou seja, escolher  os jogadores/técnico a cada rodada
para o torneio de 2018.

\section{Trabalhos Relacionados}

\section{Base de Dados}

A base de  dados \cite{git_cartola} contém as  informações sobre os
campeonatos  de 2014  a  2017 consolidadas  e  serão utilizadas  para
aprendizado do nosso  modelo, os dados sobre 2018  são adicionadas ao
término de cada rodada. No  repositório, cada campeonato é dividido
em cinco arquivos do formato \tit{csv} com as seguintes informações:
nome  dos  times,  posição  e time  dos  jogadores  (considera-se  o
técnico  um  jogador com  a  posição  de técnico),  resultado  das
partidas  do  campeonato,  lista  dos  scouts, e  a  tabela  final  da
pontuação dos times no campeonato.

\begin{table}[h]
\begin{center}
\caption[]{Tipos de Scouts}
\label{tab:model}
\begin{tabular}{| l | c | c | c | c | c | c | c | c | c | c |}
\hline
\multicolumn{2}{|c|}{Scouts de Ataque} & \multicolumn{2}{|c|}{Scouts de Defesa} \\
\hline
Gol                   &  8.0 pts & Jogo sem sofrer gols &  5.0 pts \\
Assistência           &  5.0 pts & Defesa de pênalti    &  7.0 pts \\
Finalização na trave  &  3.0 pts & Defesa difícil\ts{1} &  3.0 pts \\
Finalização defendida &  1.2 pts & Roubada de bola      &  1.5 pts \\
Finalização para fora &  0.8 pts & Gol contra           & -5.0 pts \\
Falta sofrida         &  0.5 pts & Cartão vermelho      & -5.0 pts \\
Pênalti perdido       & -4.0 pts & Cartão amarelo       & -2.0 pts \\
Impedimento           & -0.5 pts & Gol sofrido\ts{1}    & -2.0 pts \\
Passe errado          & -0.3 pts & Falta cometida       & -0.5 pts \\
\hline
\end{tabular}
\end{center}
\end{table}
% \ts{1}scots exclusivos de goleiro.

Abaixo, os  itens abaixo  complementam a \rtab{model}  com a  lista de
Scouts, também sendo atualizado a cada rodada:

\begin{itemize}

\item \textbf{atletas.nome:}                nome completo do jogador
\item \textbf{atletas.apelido:}             nome/apelido do jogador
\item \textbf{atletas.rodada\_id:}          número da rodada do Brasileirão
\item \textbf{atletas.clube\_id:}           abreviação do clube do jogador
\item \textbf{atletas.posicao\_id:}         posição do jogador
\item \textbf{atletas.clube.id.full\_name:} clube do jogador
\item \textbf{atletas.status\_id:}          status do jogador na rodada
\item \textbf{atletas.pontos\_num:}         pontuação dos scouts
\item \textbf{atletas.preco\_num:}          preço do jogador
\item \textbf{atletas.variacao\_num:}       variação do preço do jogador
\item \textbf{atletas.media\_num:}          média do jogador
\item \textbf{atletas.jogos\_num:}          quantidade de jogos disputados
\item \textbf{atletas.scout:}               quantidade de scouts obtidos

\end{itemize}

\section{Tratamento dos Dados}

Antes  de realizar  o treinamento,  uma análise  dos dados  foi feita
de  maneira  preliminar para  identificar  quais  melhorias podem  ser
realizadas, além  de eliminar inconsistências no  arquivo, utilizado
como entrada para o treinamento do modelo.


\subsection{Limpeza}

A primeira  etapa realizada  para implementar  o modelo  de predição
foi  analisar os  dados fornecidos  pela  API do  cartola, fazendo  as
limpezas necessárias para criar amostras corretas e relevantes para o
preditor. Os  dados devem  continuar consistente  após a  limpeza dos
dados, ou  seja, nenhuma  coluna poderá  ser adicionada,  alterada ou
removida. Após uma análise prévia dos dados, alguns problemas foram
detectados nas informações dos jogadores:

\begin{itemize}
  \item todos os scouts com valor NaN (\tit{not a number}).
  \item coluna 'ClubeID' com valor NaN.
  \item coluna 'Status' com valor NaN.
  \item pontuação não equivalente a soma ponderada dos scouts.
\end{itemize}

A coluna  'atletas.clube\_id' tem campos repetidos  e divergentes: por
exemplo, todos os Atléticos (MG, PR, e GO) são ATL. Além disso, há
jogadores  com  siglas diferentes  das  equipes  que eles  jogam  (por
exemplo,  Maicosuel [id:  37851]).  A coluna  'athletes.atletas.scout'
não é informativa. Os scouts de 2015 dos jogadores são cumulativos,
ou seja,  os scouts dos  jogadores vão  sendo somados a  cada rodada.
Entretanto, a pontuação não é. Isso também causa o repetimento de
dados.

As linhas que possuem as inconsistências citadas acima são removidas
ou  atualizadas. Além  destas  modificações,  outras remoções  de
dados  irrelevantes são  realizadas, por  exemplo jogadores  que não
participaram de  nenhuma rodada, técnicos e  jogadores sem posição,
jogadores sem nome, entre outros.

\subsection{Atualização dos scouts cumulativos de 2015}

Como  dito  anteriomente, os  dados  sobre  os  scouts de  2015  foram
disponibilizados  de maneira  cumulativa pela  fornecidos pela  API do
cartola, ou seja, os scouts de  uma rodada são adicionados aos scouts
anteriores a cada nova rodada  que um jogador participa. Portanto, foi
necessário tirar essa acumulação para  cada jogador, de maneira que
a representação dos dados ficasse coerente.

Para isso, dada  uma rodada específica, os scouts de  um jogador são
subtraídos  do máximo  dos scouts  de todas  as rodadas  anteriores.
Repare que assim  há chance do scout 'Jogo Sem  Sofrer Gols (SG)' ser
negativo se o jogador não sofrer  gols na rodada anterior e sofrer na
rodada atual. Quando isso acontece, esse scout é atualizado.

\subsection{Verificação da pontuação com os scouts}

Por inconsistência  na base de  dados fornecida pela API  do cartola,
alguns  jogadores possuiam  pontuações que  não condiziam  com seus
scouts. Para esses casos, o jogador  é removido da base de dados para
evitar qualquer tipo de ruído. Ao final, mais de 4000 jogadores foram
removidos.

\subsection{Remoção das linhas duplicadas}

A última operação realizada para limpar  a base de dados foi apagar
as  linhas repetidas.  A existência  de linhas  repetidas deve-se  ao
fato  de que  a partir  da primeira  participação de  um jogador  no
campeonato, ele  aparece em todas  as rodadas subsequentes,  mesmo que
não tenha jogado.  As entradas redundantes não  são necessárias ao
modelo, por isso foram removidas.

\subsection{Criação das Amostras}

Continuando  a implementação  após a  limpeza da  base de  dados, o
próximo passo  foi transformar os  dados para tornar  utilizáveis na
criação dos modelos. Desta forma, duas operações são realizadas e
descritadas abaixo:

\begin{itemize}

\item  \textbf{Selecionar somente  as colunas  de interesse:}  colunas
como  'atletas.nome', 'atletas.foto',  etc não  são relevantes  para
criação  do  modelo.  No  entanto,   colunas  como  o  'AtletaID'  e
'atletas.apelido',  mesmo  que  não utilizadas  para  treinamento  do
modelo, são importante para avaliar  o resultado e, portanto, também
serão consideradas.

\item \textbf{Converter todos os  dados categóricos para numéricos:}
as   colunas  'Posicao',   'ClubeID',  'opponent'   e  'casa'   serão
convertidas para número.

\end{itemize}

\section{Treinamento}

\subsection{Redes Neurais}

\subsection{Regressão Linear: Random Forest}

\subsection{Regressão Linear: BayesianRidge}

\subsection{Regressão Linear: Ridge}

\subsection{Regressão Linear: ElasticNet}

\subsection{Regressão Linear: Gradient Boosting}

\subsection{Regressão Linear: SVR (Support Vector Regression)}

\section{Resultados}

\subsection{Conjunto de Validação}

\subsection{Conjunto de Teste}

\section{Conclusão}

\begin{thebibliography}{00}

\bibitem{fifa} FIFA, FIFA. "FIFA/Coca-Cola World Ranking 2018."

\bibitem{news}  Martin,   William.  "Big  banks  like   Goldman  Sachs
spectacularly failed to predict the  World Cup winner — here's why".
Business Insider (2018).

\bibitem{win_cartola} GloboEsporte.com. "Com 2460.19 pontos 'Jorgito10
(O mito)' é o vencedor da liga GE AP em 2017". GloboEsporte.com.

\bibitem{viscondiaplicacao}   .VISCONDI,   et  al.   "Aplicação   de
aprendizado de  máquina para otimização  da escalação de  time no
jogo Cartola FC."

\bibitem{git_cartola}      GitHub     -      Repositório     caRtola.
\url{https://github.com/henriquepgomide/caRtola/}.

\end{thebibliography}

\end{document}
