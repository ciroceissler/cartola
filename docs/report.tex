\documentclass[conference]{IEEEtran}
\IEEEoverridecommandlockouts

\usepackage{cite}
\usepackage[portuges,brazil,english]{babel}
\usepackage{amsmath,amssymb,amsfonts}
\usepackage{algorithmic}
\usepackage{graphicx}
\usepackage{textcomp}
\usepackage[latin1,utf8]{inputenc}
\usepackage{listings}
\usepackage{color}

% Commands to help in references
\newcommand{\rchap}[1]{Chapter~\ref{chap:#1}}
\newcommand{\rsec}[1]{Section~\ref{sec:#1}}
\newcommand{\rsecs}[2]{Sections~\ref{sec:#1} --~\ref{sec:#2}}
\newcommand{\rtab}[1]{Tabela~\ref{tab:#1}}
\newcommand{\rfig}[1]{Figura~\ref{fig:#1}}
\newcommand{\rfigs}[2]{Figures~\ref{fig:#1} --~\ref{fig:#2}}
\newcommand{\rfign}[3]{Figures~\ref{fig:#1}, \ref{fig:#2} \& \ref{fig:#3}}
\newcommand{\rlst}[1]{Listing~\ref{lst:#1}}
\newcommand{\rlsts}[2]{Listing~\ref{lst:#1}~--~\ref{lst:#2}}
\newcommand{\rlstn}[3]{Listings~\ref{lst:#1}{#2}~--~\ref{lst:#1}{#3}}
\newcommand{\req}[1]{Equation~\ref{eq:#1}}
\newcommand{\reqs}[2]{Equations~\ref{eq:#1} --~\ref{eq:#2}}
\newcommand{\ttt}[1]{{\texttt{#1}}}
\newcommand{\tit}[1]{{\textit{#1}}}
\newcommand{\ts}{\textsuperscript}

\def\BibTeX{{\rm B\kern-.05em{\sc i\kern-.025em b}\kern-.08em
    T\kern-.1667em\lower.7ex\hbox{E}\kern-.125emX}}

\definecolor{codegreen}{rgb}{0,0.6,0}
\definecolor{codegray}{rgb}{0.5,0.5,0.5}
\definecolor{codepurple}{rgb}{0.58,0,0.82}
\definecolor{backcolour}{rgb}{0.95,0.95,0.92}
\definecolor{darkblue}{rgb}{0.0,0.0,0.6}

\lstdefinestyle{mystyle}{
  commentstyle=\foonotesize\color{codegreen},
  backgroundcolor=\color{backcolour},
  stringstyle=\color{codepurple},
  basicstyle=\footnotesize,
  breakatwhitespace=false,
  breaklines=true,
  captionpos=b,
  keepspaces=true,
  showspaces=false,
  showstringspaces=false,
  showtabs=false,
  tabsize=2
}

\lstset{
  classoffset=0,
  keywordstyle=\color{back},
  classoffset=1,
  morekeywords={use,hrw,module,increment,HARP,Catapult,synthesize},
  keywordstyle=\color{purple},
  classoffset=2,
  morekeywords={pragma,omp,parallel,target,map,data,device,for,},
  keywordstyle=\color{darkblue},
  frame=single,
  style=mystyle
}

\begin{document}

\title{Demistificando o Cartola FC}

\author
{
  \IEEEauthorblockN{Ciro Ceissler}
  \IEEEauthorblockA{RA 108786\\ ciro.ceissler@gmail.com}
  \and
  \IEEEauthorblockN{Lucas de Souza e Silva}
  \IEEEauthorblockA{RA \\ lucasonline1@gmail.com}
  \and
  \IEEEauthorblockN{Matheus Laborão Netto}
  \IEEEauthorblockA{RA \\ mln.laborao@gmail.com}
  \and
  \IEEEauthorblockN{Ramon Nepomuceno}
  \IEEEauthorblockA{RA 192771\\ ramonn76@gmail.com}
}

\maketitle

\section{Introduction}

O futebol é um esporte com diversos  fãs ao redor do mundo e com uma
imprevisibilidade muito grande, supreendendo os torcedores. Um exemplo
recente foi  a Copa  do Mundo de  2018 no qual  o campeão  do torneio
anterior não conseguiu passar nem da primeira fase, perdendo 2x0 para
Coreia do Sul apenas 57\ts{a}  colocada no ranking fifa \cite{fifa}, e
a presença  Croácia na  final. Ainda nesta  Copa do  Mundo; diversos
bancos, incluindo  o \tit{Goldman Sachs}, utilizaram  este evento para
demonstrar a capacidade de  prever eventos complexos \cite{news}, eles
chegaram  a rodar  milhões  variações do  torneio  para calcular  a
probabilidade de cada time avançar na competição e mesmo assim não
obtiveram um resulta satisfatório.

Os fãs  de futebol  também participam  da "brincadeira"  através do
Cartola FC, um  "fantasy game" sobre este esporte. O  Cartola FC é um
jogo online fictício no qual você pode montar seu time com jogadores
reais da Série A do Campeonato  Brasileiro. No jogo é preciso montar
seu time, escolhendo  técnico, jogadores e o esquema  tático. Com as
moedas do  jogo, inicialmente  \$ 100.00, é  possível escalar  o seu
time,  comprando  e  vendendo  jogadores  a  cada  rodada,  importante
adicionar  que a  (des)valorização  do jogador  acontece após  cada
rodada  e leva  em consideração  a pontuação  do atleta,  além da
média  dos  outros jogadores.  A  escalação  deve ser  feita  antes
do  primeiro jogo  da  rodada.  A cada  rodada,  os jogadores  recebem
pontuações  baseadas  em suas  ações  durante  as partidas,  e.g.,
gols feitos,  chutes defendidos,  bolas roubadas, faltas  cometidas ou
sofridas, cartões  recebidos, entre outras. As  ações dos jogadores
em campo são  chamadas de Scouts e são elas  que geram a pontuação
do time. Por fim, o Capitão tem sua pontuação duplicada e não pode
ser o técnico.

No  Cartola  FC   de  2017,  o  jogador   vencedor  totalizou  2460.19
pontos  ao final  de  todas  as rodadas  (média  de  67.4 pontos  por
rodada),  competindo   contra  4  milhões  de   times  escalados.  Em
\cite{viscondiaplicacao}, um sistema  de previsão aplicando técnicas
de  \tit{machine  learning}  foi  proposto  para  conseguir  a  melhor
escalação do time  para a próxima rodada,  entretanto os resultados
obtidos não permitiria  vencer o torneio. Além  disso, este trabalho
fez uma análise  dos dados, explanando algumas  observações como: o
esquema  tático tem  uma  leve  variação, sendo  o  4-4-2 a  melhor
formação.

O  projeto  proposto propõe  um  modelo  preditivo para  maximizar  a
pontuação  do Cartola  FC, informando  a cada  rodada do  jogo quais
parâmetros, escolher  os jogadores  a cada  rodada, que  deverão ser
utilizado para o torneio de 2018.

\section{Base de Dados}

A base de  dados \cite{git_cartola} contém as  informações sobre os
campeonatos  de 2014  a  2017 consolidadas  e  serão utilizadas  para
aprendizado do nosso  modelo, os dados sobre 2018  são adicionadas ao
término de cada rodada. No  repositório, cada campeonato é dividido
em cinco arquivos do formato \tit{csv} com as seguintes informações:
nome  dos  times,  posição  e time  dos  jogadores  (considera-se  o
técnico  um  jogador com  a  posição  de técnico),  resultado  das
partidas  do  campeonato,  lista  dos  scouts, e  a  tabela  final  da
pontuação dos times no campeonato.

\begin{table}[h]
\begin{center}
\caption[]{Tipos de Scouts}
\label{tab:model}
\begin{tabular}{| l | c | c | c | c | c | c | c | c | c | c |}
\hline
\multicolumn{2}{|c|}{Scouts de Ataque} & \multicolumn{2}{|c|}{Scouts de Defesa} \\
\hline
Gol                   &  8.0 pts & Jogo sem sofrer gols &  5.0 pts \\
Assistência           &  5.0 pts & Defesa de pênalti    &  7.0 pts \\
Finalização na trave  &  3.0 pts & Defesa difícil\ts{1} &  3.0 pts \\
Finalização defendida &  1.2 pts & Roubada de bola      &  1.5 pts \\
Finalização para fora &  0.8 pts & Gol contra           & -5.0 pts \\
Falta sofrida         &  0.5 pts & Cartão vermelho      & -5.0 pts \\
Pênalti perdido       & -4.0 pts & Cartão amarelo       & -2.0 pts \\
Impedimento           & -0.5 pts & Gol sofrido\ts{1}    & -2.0 pts \\
Passe errado          & -0.3 pts & Falta cometida       & -0.5 pts \\
\hline
\end{tabular}
\end{center}
\end{table}
% \ts{1}scots exclusivos de goleiro.

Abaixo, os itens que complementam o arquivo \tit{csv} com a lista
de Scouts, sendo atualizado a cada rodada:

\begin{itemize}

\item \textbf{atletas.nome:}                nome completo do jogador
\item \textbf{atletas.apelido:}             nome/apelido do jogador
\item \textbf{atletas.rodada\_id:}          número da rodada do Brasileirão
\item \textbf{atletas.clube\_id:}           abreviação do clube do jogador
\item \textbf{atletas.posicao\_id:}         posição do jogador
\item \textbf{atletas.clube.id.full\_name:} clube do jogador
\item \textbf{atletas.status\_id:}          status do jogador na rodada
\item \textbf{atletas.pontos\_num:}         pontuação dos scouts
\item \textbf{atletas.preco\_num:}          preço do jogador
\item \textbf{atletas.variacao\_num:}       variação do preço do jogador
\item \textbf{atletas.media\_num:}          média do jogador
\item \textbf{atletas.jogos\_num:}          quantidade de jogos disputados
\item \textbf{atletas.scout:}               quantidade de scouts obtidos

\end{itemize}

\begin{thebibliography}{00}

\bibitem{fifa} FIFA, FIFA. "FIFA/Coca-Cola World Ranking 2018."

\bibitem{news}  Martin,   William.  "Big  banks  like   Goldman  Sachs
spectacularly failed to predict the  World Cup winner — here's why".
Business Insider (2018).

\bibitem{win_cartola} GloboEsporte.com. "Com 2460.19 pontos 'Jorgito10
(O mito)' é o vencedor da liga GE AP em 2017". GloboEsporte.com.

\bibitem{viscondiaplicacao}   .VISCONDI,   et  al.   "Aplicação   de
aprendizado de  máquina para otimização  da escalação de  time no
jogo Cartola FC."

\bibitem{git_cartola}      GitHub     -      Repositório     caRtola.
\url{https://github.com/henriquepgomide/caRtola/}.

\end{thebibliography}

\end{document}
